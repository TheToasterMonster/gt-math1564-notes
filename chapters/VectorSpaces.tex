\chapter{Vector Spaces}

\section{Motivation}
The usual type of vector that we're familiar with is
of the form $(x_1, \dots, x_n) \in \R^n$. In this
way, we can add two vectors with an addition
operation\footnote{The notation $f : A \to B$ means that
$f$ is a function from $A$ to $B$, i.e. it has domain $A$ and codomain $B$.} $+ : \R^n \times \R^n \to \R^n$ and
multiply a vector by a scalar (from $\R$)
with a scalar multiplication operation
$\cdot : \R \times \R^n \to \R^n$, both defined
component-wise. We would like to generalize this
structure to a more abstract setting.

\section{Abstract Vector Spaces}

\begin{tcolorbox}
  \begin{definition}[Vector space]
    A \emph{vector space}
    $(V, +_v, \cdot_v,)_K$ over a field\footnote{We call $K$ the \emph{base field} of $V$.} $K$ is a set
    $V$ equipped with a binary addition operation
    $+_v : V \times V \to V$ and a scalar multiplication\footnote{We will sometimes just write $\alpha x$ instead of $\alpha \cdot_v x$ in these axioms for notational clarity.}
    operation $\cdot_v : K \times V \to V$
    that satisfy the following 10 axioms:
    \begin{enumerate}[(i)]
      \item \emph{Closure under addition}:
        For all $x, y \in V$, $x +_v y \in V$.
      \item \emph{Additive commutativity}:
        For all $x, y \in V$, $x +_v y = y +_v x$.
      \item \emph{Additivity associativity}:
        For all $x, y, z \in V$,
        $(x +_v y) +_v z = x +_v (y +_v z)$.
      \item \emph{Existence of additive identity}:
        There exists $0_v \in V$ such that
        $0_v +_v x = x$ for all $x \in V$.
      \item \emph{Existence of additive inverse}:
        For all $x \in V$, there exists $(-x) \in V$
        such that $x +_v (-x) = 0_v$.
      \item \emph{Closure under scalar multiplication}:
        For all $\alpha \in K$ and $x \in V$,
        $\alpha \cdot_v x \in V$.
      \item \emph{Distributive property with scalars}:
        For all $\alpha \in K$ and $x, y \in V$,
        $\alpha (x +_v y) = \alpha x +_v \alpha y$.
      \item \emph{Multiplicative associativity}:
        For all $\alpha, \beta \in K$ and $x \in V$,
        $\alpha(\beta x) = (\alpha \beta) x$.
      \item \emph{Distributive property with vectors}:
        For all $\alpha, \beta \in K$ and $x \in V$,
        $(\alpha + \beta) x = \alpha x +_v \beta x$.
      \item \emph{Existence of multiplicative identity}:
        There exists $1 \in K$ such that
        $1 \cdot_v x = x$ for all $x \in V$.
    \end{enumerate}
  \end{definition}
\end{tcolorbox}

Note that we will often omit the subscripts on $+_v$
and $\cdot_v$ when the vector space is clear from context.
We often also just write $V$ to refer to
$(V, +_v, \cdot_v)_K$ when the base field and
operations are clear from context.
We sometimes say that a vector space is \emph{abstract}
if it is not $\R^n$, which we sometimes refer to as
\emph{concrete}. It is not too important in our context
to define what a field is, we will almost always
take $K = \R$ or $\C$.

Just from these axioms, we can already prove some
immediate (perhaps obvious) properties.

\begin{tcolorbox}
  \begin{prop}
    Let $V$ be a vector space and $0_v \in V$ be
    its additive identity. Then
    $0 \cdot x = 0_v$ for any $x \in V$.
  \end{prop}
\end{tcolorbox}

\begin{proof}
  By the distributive property, we have
  \[
    0 x = (0 + 0) x = 0 x + 0 x.
  \]
  Adding the additive inverse of $0 x$ to both sides,
  we obtain $0_v = 0x$ as desired.
\end{proof}

\begin{tcolorbox}
  \begin{prop}
    The additive identity of a vector space is unique.
  \end{prop}
\end{tcolorbox}

\begin{proof}
  Let $V$ be a vector space and $0_v, \widetilde{0}_v \in V$
  both be additive identities. Then
  \[
    0_v = \widetilde{0}_v + 0_v = 0_v + \widetilde{0}_v = \widetilde{0}_v
  \]
  by commutativity and the additive identity property.
  So the additive identity is unique.
\end{proof}
