\chapter{Elementary Set Theory}

\section{Introduction to Sets}
Informally, a \emph{set} is a collection of things.
We can write a set by enumerating its elements as
\[A = \{1, 2, 3\},\]
or we can simply describe its elements by writing
\[B = \{x \mid x \text{ is a real number}\}.\]
The ``$\mid$'' symbol is read as ``such that,''
and the colon ``:'' is sometimes used to mean the
same thing. Some commonly used sets of numbers are:
\begin{itemize}
  \item $\R$, the set of real numbers,
  \item $\Q$, the set of rational numbers,
  \item $\C$, the set of complex numbers,
  \item $\Z$, the set of integers,
  \item and $\N$, the set of natural numbers.
\end{itemize}

\section{Set Membership and Subsets}
We write $x \in A$ to mean that $x$ is an element of $A$
and $x \notin A$ otherwise.
For example,
\[
  1 \in \{1, 2, 3\}, \quad
  4 \notin \{1, 2, 3\}, \quad
  \text{and} \quad
  \pi \notin \Z.
\]
We write $\varnothing$ to denote the empty set, and when
the scope of a problem is understood,
we write $\mathcal{U}$ to denote the universal set.
In other words, $x \notin \varnothing$ and
$x \in \mathcal{U}$ for any $x$.

For two sets $A$ and $B$, we say that $A$ is a
\emph{subset} of $B$ if every element of $A$ is also
an element of $B$. We write $A \subseteq B$ or
$A \subset B$ when this is the case. For example,
\[\{1\} \notin \{1, 2, 3\} \quad \text{but} \quad \{1\} \subseteq \{1, 2, 3\}.\]
Note that $\varnothing \subseteq A$ for any set $A$
(also $\varnothing \subseteq \varnothing$ but
$\varnothing \notin \varnothing$).

\emph{Equality} for sets is defined in terms of subsets. We
say that $A = B$ if $A \subseteq B$ and $B \subseteq A$.
Note that sets are inherently unordered and contain
unique elements, so that
\[\{1, 2, 3\} = \{3, 2, 1\} \quad \text{and} \quad \{1, 2, 2, 3\} = \{1, 2, 3\}.\]
We write
$A \subsetneq B$ when $A$ is a \emph{proper} subset of
$B$, i.e., $A \subseteq B$ but $A \neq B$.

\section{The Cartesian Product}
For two sets $A$ and $B$, their \emph{Cartesian product}
is defined as
\[A \times B = \{(x, y) \mid x \in A, y \in B\}.\]
This is technically not a set operation, but rather
shorthand for the description of a new set. We can also
define the Cartesian product of $n$ sets
$A_1, \dots, A_n$ as
\[
  A_1 \times \cdots \times A_n
  = \{(x_1, \dots, x_n) \mid x_1 \in A_1, \dots, x_n \in A_n\}.
\]
Note that by this definition,
$A \times B \times C \ne A \times (B \times C)$.
We also write $A^n$ to mean $A \times \cdots \times A$
($n$ times).
As an example, the set $\R^2 = \R \times \R$ is the
usual 2-D Cartesian plane.

\section{Set Operations}
Let $A$ and $B$ be two sets. We can define the following
operations on $A$ and $B$:
\begin{itemize}
  \item the \emph{union} $A \cup B = \{x \mid x \in A \text{ or } x \in B\}$,
  \item the \emph{intersection} $A \cap B = \{x \mid x \in A \text{ and } x \in B\}$,
  \item the \emph{set difference} $A \setminus B = \{x \mid x \in A \text{ and } x \notin B\}$,
    we sometimes read this as ``$A$ take away $B$,''
  \item and when a universal set $\mathcal{U}$ is
    understood, the \emph{complement}
    $A^c = \mathcal{U} \setminus A$.
\end{itemize}
For example, if $\mathcal{U} = \R^2$ and $A = \{(x, y) \mid x^2 + y^2 < 1\}$,
then $A^c = \{(x, y) \mid x^2 + y^2 \geq 1\}$, i.e.
the set of all points outside the unit circle and on
its edge.

\begin{tcolorbox}
\begin{prop}
  For any two sets $A$ and $B$, we have
  $A = (A \cap B) \cup (A \setminus B)$.
\end{prop}
\end{tcolorbox}

\begin{proof}
  We need to show that $\text{LHS} \subseteq \text{RHS}$
  and $\text{RHS} \subseteq \text{LHS}$.

  To see that $\text{LHS} \subseteq \text{RHS}$,
  take any element $x \in \text{LHS} = A$. If
  $x \in A \setminus B$, then
  $x \in (A \cap B) \cup (A \setminus B)$ and we're done.
  Otherwise $x \notin A \setminus B$ and so
  $x \in B$ since we assumed that $x \in A$.
  Then $x \in A \cap B$ and thus
  $x \in (A \cap B) \cup (A \setminus B)$. So
  $x \in \text{RHS}$ and thus
  $\text{LHS} \subseteq \text{RHS}$
  as desired.

  For $\text{RHS} \subseteq \text{LHS}$,
  take any element $x \in \text{RHS} = (A \cap B) \cup (A \setminus B)$.
  If $x \in A \setminus B$, then $x \in A$ by definition.
  Otherwise $x \in A \cap B$ and so $x \in A$ also.
  Thus $x \in \text{LHS}$ and $\text{RHS} \subseteq \text{LHS}$ as desired.
\end{proof}

\begin{remark}
  This is related to the \emph{law of
  total probability}, which states that
  for any two events $A$ and $B$,
  \[
    \Pr(A)
    = \Pr(A \cap B) + \Pr(A \cap B^c)
    = \Pr(A | B)\Pr(B) + \Pr(A | B^c)\Pr(B^c).
  \]
\end{remark}

\section{Set Laws}
The \emph{distributive laws} say that
for any three sets $A$, $B$, and $C$,
\[
  A \cup (B \cap C) = (A \cup B) \cap (A \cup C)
  \quad \text{and} \quad
  A \cap (B \cup C) = (A \cap B) \cup (A \cap C).
\]
The so-called \emph{de Morgan's laws} say that
for any two sets $A$ and $B$,
\[
  (A \cup B)^c = A^c \cap B^c
  \quad \text{and} \quad
  (A \cap B)^c = A^c \cup B^c.
\]
Notice that de Morgan's laws allow us to switch
from unions to intersections and vice versa.
